\documentclass{beamer}
%\usecolortheme{seahorse} % change this
\usepackage{graphicx}
\usepackage{inconsolata}
\usepackage{anyfontsize}
\usepackage{courier}
\usepackage{color}


\usepackage{natbib} 
%bibstyle muss einer da sein, 
% setimmt wie der kram an \cite aussieht.
% https://de.wikibooks.org/wiki/LaTeX-W%C3%B6rterbuch:_bibliographystyle
%https://de.sharelatex.com/learn/Bibtex_bibliography_styles
%\bibliographystyle{abbrvnat} 
%\bibliographystyle{unsrt} 
\bibliographystyle{plainnat}

%gets rid of bottom navigation bars
\setbeamertemplate{footline}[frame number]{}

%gets rid of bottom navigation symbols
\setbeamertemplate{navigation symbols}{}

%gets rid of footer
%will override 'frame number' instruction above
%comment out to revert to previous/default definitions
\setbeamertemplate{footline}{}

\newcommand{\red}[1]{\textcolor{red}{#1}}

\title 
{Generative adversarial training on \\ structured domains}
\author % NEEDS MOAR INFO ,, contact etc
%{Stefan Mautner }
{\underline{Stefan Mautner} \and Fabrizio Costa 
    \small{ 
        \texttt{
            \href{mailto:mautner@informatik.uni-freiburg.de}
            {mautner@informatik.uni-freiburg.de}
        }
        \texttt{
            \href{mailto:f.costa@exeter.ac.uk}
            {f.costa@exeter.ac.uk}
        }
   }
}

\date 
{Freiburg University \\2017-02-16}

\titlegraphic{\includegraphics[width=2cm]{images/logo.jpg}
}

\begin{document}
\frame{\titlepage}



% design goals
\begin{frame}
\frametitle{Constructive Machine Learning}

    \begin{itemize}
        \item {\bf What:} answer \red{design} questions using examples
        \item We are interested in: \\
        constructive approaches for \red{structured} domains
        \item In chemo- and bio-informatics: \\
        synthesize molecules with a desired bio-activity
    \end{itemize}
    \begin{figure}
        \centering
        \includegraphics[width=0.5\textwidth]{images/mol.jpg}
        \includegraphics[width=0.5\textwidth]{images/rna.png}
    \end{figure}    
\end{frame}




%%%%%%%%%%%%%%%%%%%%%%%%%%%%%%%%%%%%%%%%%%%%%%%%%%%%%%%%%%%%%%%%%%%%%%%%%%%%
% add iverview slide... 

\begin{frame}
\frametitle{Assumed work}
    \begin{itemize}
        \item EDeN (Explicit Decomposition with Neighborhoods)\footnote{github.com/fabriziocosta/EDeN}
        \begin{itemize}
            \item Vectorizes Graphs
            \item Used when training model from graphs
            \item (not discussed here)
        \end{itemize}    
    
    \item GraphLearn \footnote{github.com/fabriziocosta/GraphLearn} 
        \begin{itemize}
            \item generates instances given examples
            \item (overview given here)
        \end{itemize}    
        \item This is a sampling extension for GraphLearn
    \end{itemize}    

\end{frame}
%%%%%%%%%%%%%%%%%%%%%%%%%%%%%%%%%%%%%%%%
\begin{frame}
\frametitle{The Problem}
    \begin{itemize}
        \item Density estimation based on observed graphs (preferably few, "positive"
            class only)
        \item Implies loose constraints on feasible manyfold
        \item \red{Question}: How tighten constraints?
            %a: Generative adversarial training
    \end{itemize}
    \begin{figure}[ht]
        \centering
        \footnote{ Yann Le Cun, NIPS 2016 Keynote (modified)}
        \includegraphics[width=0.8\textwidth]{images/valley.png}
    \end{figure}    
\end{frame}



\begin{frame}
\frametitle{Artificial neural networks}
    \begin{itemize}

        \item ANNs also try to solve creative tasks, 
        Popular are: \footnote{Ian Goodfellow, NIPS 2016 Tutorial }
        \item Fully visible belief networks (FVBNs)
        \item Variational autoencoders
        \item \red{Generative adversarial networks GANs}

            %\red{ === gans developed in this comunity .. generative pictures}
    \end{itemize}

    %\begin{itemize}
    %    \item say something about gans
    %\end{itemize}

    \begin{figure}[ht]
        \centering
        \footnote{Al Gharakhanian,  Blogpost, Dec 2016}
        \includegraphics[width=0.5\textwidth]{images/GAN.png}
    \end{figure}   
\end{frame}



\begin{frame}

\frametitle{GAN examples}
    \begin{figure}[ht]
        \centering
        \footnote{Legid \emph{et al.} 2016}
        \includegraphics[width=0.7\textwidth]{images/GAN2.png}
    \end{figure}   
    \begin{figure}[ht]
        \centering
        \footnote{github.com/mattya/chainer-DCGAN }
        \includegraphics[width=0.7\textwidth]{images/animee.png}
    \end{figure}   
\end{frame}

\begin{frame}
\frametitle{GAT on structured domains}
    \begin{itemize}
            % why the new thing should work etc then proposal.
            % comes from NN etc now we use GAN for structures

            % shpere with data drawing... regions where dots are dense and undense
            % but 1class says things are ok if above plane..
            % arrow to undense region -> point to bad graphs
        %\item To generate instances, \red{we} require a model 
        %\item There might not be enough train data to capture the densities correctly. 
        \item The graph generation guiding model might be too permissive
        \item ANN researchers inspired us by addressing  a very similar problem using GANs
        %\item use gen instances to train better discriminators
        %\item use made graphs and make increatingly good graphs 4 graphs
        
        \item {\bf Proposal:} Generate instances and assume they are negative examples
            to \red{improve} the generation guiding model
    \end{itemize}
    \begin{figure}[ht]
        \centering
        \includegraphics[width=0.7\textwidth]{images/valley_x3_stars.png}
    \end{figure}    
\end{frame}

%%%%%%%%%%%%%%%%%%%%%%%%%%%%%%%%%%%%%%%%%%%%%%%%%%%%%%%%%%%%%%%%%%%%%%%%%%%%

\begin{frame}
\frametitle{The constructive learning problem for finite samples\footnote{Costa Artif. Intell. 2016}}
    \begin{itemize}
        \item Given a set of graphs $G$
        \item use a parametrized generator $M_\theta$ to \red{construct} set $G_\theta$
        \item find optimal $\theta$ to \red{jointly} satisfy:
            \begin{enumerate}
        \item probability density is the \red{same} if estimated over $G$ or $G_\theta$
        \item $G_\theta$ \red{differs} from $G$
            \end{enumerate}
        \item Optimize:\\ \begin{center} $\arg \min_\theta L(P(G),P(G_\theta)) + \lambda  ~ Sim(G, G_\theta)$ \end{center}
        \item where:
        \begin{itemize}
            \item $L$ is a loss over probability distributions \\(e.g. symmetric Kullback Leibler divergence)
            \item $Sim$ is a {\em set} graph kernel
            \item $\lambda$ is desired trade off
        \end{itemize}
    
    \end{itemize}
    %$${argmin}_{\theta}~~L(f_{G_0}(G),f_{G_{\theta}}(G))+\lambda \frac{K(G_0,G_{\theta})}{\sqrt{K(G_0,G_0),K(G_{\theta},G_{\theta})}}}$$
\end{frame}




% so far we have the basic graphlearn
\begin{frame}
    \frametitle{Parametrized Generator for Graphs}
    \begin{itemize}
        \item Instead of generating $\mapsto$ \red{sample} from a corresponding probability distribution over graphs
        \item We use Metropolis Hastings (MH) \begin{tiny}Markov Chain Monte Carlo (MCMC)\end{tiny}
        \begin{enumerate}
            \item start from {\em seed} graph $x$
            \item propose according to $g(x \mapsto x')$
            \item accept according to: \\
            \begin{center}
            $A(x \mapsto x')=\min(1, \frac{P(x')}{P(x)} \frac{g(x \mapsto x')}{g(x' \mapsto x)})$
            \end{center}
        \end{enumerate}
        \item {\bf Q:} how not to reject proposed graphs too often?
        \item {\bf A:} use graph \red{grammar} induced from data  for $g(x \mapsto x')$
    \end{itemize}
\end{frame}

\begin{frame}
    \frametitle{Graph Grammar}
    A graph grammar is a finite set of productions P=(M,D,E) 
    \begin{itemize}
        \item M=mother graph
        \item D=daughter graph
        \item E=embedding mechanism
    \end{itemize}
    \begin{figure}[ht]
        \centering
        \includegraphics[width=0.9\textwidth]{images/grammar.pdf}
    \end{figure}
\end{frame}

\begin{frame}
    \frametitle{Substitutable Graph Grammar}
    \begin{itemize}
        \item \red{cores} (neighborhood graphs) can be substituted..
        \item .. if they have the same \red{interface} graph
    \end{itemize}
    \begin{figure}[ht]
        \centering
        \includegraphics[width=0.9\textwidth]{images/cip1.pdf}
    \end{figure}
\end{frame}

\begin{frame}
    \frametitle{Substitutable Graph Grammar}
    \begin{itemize}
        \item \red{cores} (neighborhood graphs) can be substituted 
..
        \item .. if they have the same \red{interface} graph
    \end{itemize}
    \begin{figure}[ht]
        \centering
        \includegraphics[width=0.9\textwidth]{images/cip2.pdf}
    \end{figure}
\end{frame}



% write about ze new algo :)

\begin{frame}
    \frametitle{Generative adversarial training}
    Input: \emph{train}; a set of observed  instances
    \begin{enumerate}
            % IMPROVE THIS name of sets need change
            % also show that previous gens are used

        \item train \red{one class} model on \emph{train}
        \item use \emph{train} as seeds for generation
        \item train \red{two class} model on \emph{train} (classlabel 1) \\
            and all generated instances (classlabel 0)
        \item use \emph{train} as seeds for generation
        \item goto 3
    \end{enumerate}
    \begin{figure}[ht]
        \centering
        \includegraphics[width=0.5\textwidth]{images/genmod_recol.png}
    \end{figure}
\end{frame}


% show that each generation the graphs get betta?

% BENCHMARK II
\begin{frame}
    \frametitle{Training accuracy on internal models}
    \begin{itemize}
        \item Are generated instances similar to the observed instances?
    \end{itemize}

   \begin{figure}[ht]
        \centering
        \includegraphics[width=0.70\textwidth]{images/eval2.png}
        \footnote{500 graphs in train set, 3 repeats, pubchem aid 651610}
    \end{figure}
   \small{\em lower accuracy indicates that the sets are becoming harder to separate}
\end{frame}

% BENCHMARK I 
\begin{frame}
    \frametitle{Test accuracy on internal model}
    \begin{itemize}
            % new headlines for all pictures
            % fir linear regressor  or quadratic
            % make first and last datapoints visible 
        \item Is the observed class actually learned?
            \footnote{note that the training process has never seen a real negative instance}
        \item Lower training accuracy coincides with higher test accuracy
    \end{itemize}

   \begin{figure}[ht]
        \centering
        \includegraphics[width=0.70\textwidth]{images/eval3.png}
    \end{figure}
   %\small{\em scores above horizontal dotted line indicate reliable identification}
\end{frame}



\begin{frame}
    \frametitle{Conclusion}
    
    \begin{itemize}
        \item  Generative adversarial training effective in tightening 
            generation contrains
        \item Test on larger data set required
        \item What happens when real negatives are provided in the first pass?
    \end{itemize}

    %\begin{itemize}
    %\end{itemize}
\end{frame}



% OWARI DA 
\begin{frame}
    \frametitle{The End}
    
    \begin{itemize}
        \item  Thank you
        %\item {\bf Future work:} \\\red{learn} task specific coarsening strategy directly from data   
    \end{itemize}

    %\begin{itemize}
    %\end{itemize}
\end{frame}


% BENCHMARK III
%\begin{frame}
%    \frametitle{Evaluation III}
%    \begin{itemize}
%        \item Objective quality of designs \red{THIS IS STILL BAD}
%    \end{itemize}
%
%   \begin{figure}[ht]
%        \centering
%        \includegraphics[width=0.80\textwidth]{images/eval1.png}
%    \end{figure}
%   %\small{\em scores above horizontal dotted line indicate reliable identification}
%\end{frame}



\end{document}
